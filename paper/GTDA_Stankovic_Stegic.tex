% !TEX encoding = UTF-8 Unicode

\documentclass[a4paper]{article}

\usepackage{color}
\usepackage{url}
%\usepackage[T2A]{fontenc} % enable Cyrillic fonts
\usepackage[utf8]{inputenc} % make weird characters work
\usepackage{graphicx}

\usepackage[english,serbian]{babel}
%\usepackage[english,serbianc]{babel} %ukljuciti babel sa ovim opcijama, umesto gornjim, ukoliko se koristi cirilica
\setcounter{tocdepth}{1}
\usepackage[unicode]{hyperref}
\hypersetup{colorlinks,citecolor=green,filecolor=green,linkcolor=blue,urlcolor=blue}
\usepackage{amsmath}
%\newtheorem{primer}{Пример}[section] %ćirilični primer
\newtheorem{primer}{Primer}[section]
\newtheorem{definicija}[primer]{Definicija}

\graphicspath{{./slike/}}

\begin{document}

\title{Analiza podataka o globalnom terorizmu \\ \small{~\\Seminarski rad u okviru kursa\\Istraživanje podataka\\ Matematički fakultet}}

\author{
	Una Stanković, Uroš Stegić\\
	una\_stankovic@yahoo.com, urosstegic@gmx.com}
\date{16.~maj 2017.}
\maketitle

\abstract{U radu će biti predstavljena analiza podataka o globalnom terorizmu dobijena primenom metoda i tehnika obrađenih na kursu Istraživanje podataka.}
	

\tableofcontents

\newpage


\section{Uvod}
\label{sec:uvod}
Terorizam je pojam koji se u poslednjih 50 godina sve češće javlja,a posebno u formi globalni terorizam, zato što je terorizam pojava koja je zahvatila ceo svet.\\\\ Prema jednoj od definicija \textit{terorizam je smišljena upotreba nezakonitog nasilja ili pretnje nasiljem radi usađivanja straha, sa namerom prisiljavanja ili zastrašivanja vlasti ili društva, kako bi se postigli najčešće politički verski ili ideološki ciljevi.} \\\\
Posmatrajući globalni nivo, ne postoji nijedna država na svetu koja je imuna na terorizam. On je poput bolesti koja počne iz jednog centra, a potom se širi na sve veće i veće oblasti. U početku je terorizam bio motivisan težnjama ka slobodi, borbama za nacionalno oslobođenje, borbi protiv režima, ali, vremenom, terorizam je poprimio nove, mnogo brutalnije i opasnije oblike.\\
Jedna od najvećih opasnosti terorizma, i jedan od razloga zašto je terorizam veoma opaka $"$bolest$"$, leži u tome što on direktno ugrožava unutrašnju i međunarodnu bezbednost, ali kao najgoru posledicu od svih, on ugrožava bezbednost pojedinca. Posmatrajući terorizam makar i u samo poslednjih nekoliko godina, iako on ni po čemu nije novost, jer se, u relativno savremenom obliku, javio još 1950-ih godina, možemo uočiti da je zahvatio ceo svet i da više niko ne može da se oseća apsolutno bezbedno. Teroristima više nisu glavne mete državne zgrade, zvaničnici ili velike kompanije, već su mete što širi dijapazon ljudi, bez obzira na poreklo, profesiju, religiju ili rasu. Poruka koju teroristi današnjice žele da pošalju je - niko nije bezbedan. \\\\
Terorizam ima veliki uticaj u međunarodnim odnosima i ekonomiji, a poslednjih nekoliko godina aktivno utiče na promenu demografske slike Evrope i izaziva kontroverzne i, često, nehumane reakcije mnogih država Evropske unije, kao i stvaranje neonacističkih i ultranacionalističkih pokreta koji se aktivno bore protiv imigranata. Osim ovakvih grupa, aktivan je porast i u broju terorističkih organizacija i njihovih članova, koji popularizacijom, dostupnošću pristupa i anonimnosti na internetu uspevaju da dopru do novih, sve mlađih, članova.\\ Upravo zbog ogromnog uticaja koji terorizam ima na stvaranje novog svetskog poretka veoma je važno vršiti analizu podataka o terorističkim napadima, kako bi se došlo do zaključaka u kom pravcu se terorizam kreće, koliku štetu nanosi, kakve promene donosi, koliko života odnosi i kako će se u budućnosti razvijati, sa ciljem da se budući napadi lakše predvide, preduprede i spreče, ili da se barem broj budućih žrtava svede na minimum.



\section{Zaključak}
\label{sec:zakljucak}
Na osnovu podataka koje smo dobili analizom, došli smo do sledećih zaključaka.
...
Veliki trud se ulaže u smanjenje globalnog terorizma i nadamo se da će upravo analiza podataka biti od velikog značaja u sprečavanju terorizma u budućnosti i pomoći da se spasu desetine hiljada ljudskih života koji se svake godine ugase isključivo zbog terorizma.

\addcontentsline{toc}{section}{Literatura}
\appendix
\bibliography{seminarski} 
\bibliographystyle{plain}


\end{document}
